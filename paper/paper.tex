\documentclass{sig-alternate}
\usepackage{algorithm,algpseudocode}

\begin{document}

\conferenceinfo{HPDC}{'12 Delft, The Netherlands}
%\CopyrightYear{2007} % Allows default copyright year (20XX) to be over-ridden - IF NEED BE.
%\crdata{0-12345-67-8/90/01}  % Allows default copyright data (0-89791-88-6/97/05) to be over-ridden - IF NEED BE.

\title{Cost- and Deadline-constrained Scheduling of Scientific Workflow Ensembles in IaaS Clouds}
%\subtitle{[Extended Abstract]}

\numberofauthors{2}
\author{
    \alignauthor Maciej Malawski and Jarek Nabrzyski\\
       \affaddr{University of Notre Dame}\\
       \affaddr{111 Information Technology Center}\\
       \affaddr{Notre Dame, IN, USA}\\
       \email{\{mmalawsk,naber\}@nd.edu}
    \alignauthor Gideon Juve and Ewa Deelman\\
       \affaddr{USC Information Sciences Institute}\\
       \affaddr{4676 Admiralty Way}\\
       \affaddr{Marina del Rey, CA, USA}\\
       \email{\{gideon,deelman\}@isi.edu}
}

\maketitle

\begin{abstract}
Abstract goes here.
\end{abstract}

%\category{H.4}{Information Systems Applications}{Miscellaneous}
%\category{D.2.8}{Software Engineering}{Metrics}[complexity measures, performance measures]

%\terms{Theory}

\keywords{Scientific workflows, DAG scheduling, simulation}

\section{Introduction}


\section{Related Work}
General policy and rule-based approaches to dynamic provisioning (e.g. Amazon Auto Scaling\footnote{http://aws.amazon.com/autoscaling} and RightScale\footnote{http://www.rightscale.com}).

Policy-based approaches for scientific workloads (e.g. \cite{Marshall2010, Kim2011}). Our approach is different in that we consider workflows, while policy based approaches typically consider bags of independent tasks or unpredictable batch workloads. This enables us to take advantage of scheduling heuristics that cannot be applied to independent tasks.

Deadline-constrained cost-minimization workflow scheduling. Our work is different from \cite{Yu2005, Abrishami2010} in that we consider ensembles of workflows in IaaS clouds, which allow one to provision resources on a per-hour billing model, rather than utility grids, which allow one to choose from a pool of existing resources with a per-job billing model. Our work is different from \cite{Mao2011} in that we consider ensembles of workflows rather than unpredictable workloads containing workflows. We also have budget constraints rather than cost minimization as a goal. In other words, we assume that there is more work to be done than the available budget, so some work must be rejected. Therefore cost is not something we optimize, but rather a constraint.

Budget-constrained workflow scheduling \cite{Sakellariou2007}.

Bi-criteria scheduling and multi-criteria scheduling. These approaches are similar to ours in that we have two scheduling criteria: cost and makespan. The challenge in multi-criteria scheduling is to derive an objective function that takes into account all of the criteria. These approaches typically use metaheuristics that run for a long time before producing good results.

\section{Problem Description}
\subsection{Application Model}
\subsection{System Model}


\section{Algorithms}
\subsection{Dynamic Provisioning Dynamic Scheduling (DPDS)}
\subsection{Workflow-Aware DPDS (WA-DPDS)}
\subsection{Static Provisioning Static Scheduling (SPSS)}

% This is an example of how we can do algorithms
\begin{algorithm}
\caption{Admit DAGs}
\label{alg:admit}
\begin{algorithmic}[1]
\Procedure{AdmitDAGs}{$W,b,d$}
    \State $P\gets empty\ plan$
    \State $A\gets \emptyset$ \Comment{Set of admitted DAGs}
    \For{$w\ in\ W$}
        \State $P^\prime \gets$\ \Call{PlanDAG}{$w,P,d$}
        \If{$P^\prime\ is\ a\ valid\ plan$}
            \If{$Cost(P^\prime) \le b$}
                \State $P\gets\ P^\prime$ \Comment{Accept new plan}
                \State $A \gets A\ +\ w$ \Comment{Admit w}
            \EndIf
        \EndIf
    \EndFor
    \State \textbf{return} $P,A$
\EndProcedure
\end{algorithmic} 
\end{algorithm}


\begin{algorithm}
\caption{PlanDAG}
\label{alg:plandag}
\begin{algorithmic}[1]
\Procedure{PlanDAG}{$w,P,d$}
    \State $P^\prime\gets \emptyset$
    \State assign each task in w to its cheapest resource type
    \State $cp \gets\ CriticalPath(w)$
    \While{$cp > d$}
        \State upgrade resource type of task with greatest improvement in runtime for least cost
        \State $cp \gets\ CriticalPath(w)$
    \EndWhile
    \State \Call{DeadlineDistribution}{w,d}
    \For{$t\ in\ w\ sorted\ by\ Deadline(t)$}
        \State choose resource r in $P^\prime$ that minimizes Cost(t)
        \If{$AFT(t,r) < Deadline(t)$}
            \State Schedule(t,r)
        \Else
            \State provision a new resource r
            \State Schedule(t,r)
        \EndIf
    \EndFor
    \State \textbf{return} $P^\prime$
\EndProcedure
\end{algorithmic} 
\end{algorithm}


\begin{algorithm}
\caption{Deadline Distribution}
\label{alg:deadlinedistribution}
\begin{algorithmic}[1]
\Procedure{DeadlineDistribution}{$w,d$}
    \State assign $Level(t)$ to each task t in w
    \For{$level\ l\ in\ w$}
        \State $Tasks(l) \gets$ number of tasks in l
        \State $Runtime(l) \gets$ runtime of tasks in l
    \EndFor
    \State $T \gets$ total number of tasks in w
    \State $R \gets$ total runtime of tasks in w
    \State $CP \gets$ critical path of w
    \For{$level\ l\ in\ w$}
        \State $r \gets \alpha * (Runtime(l)/R)$
        \State $t \gets (1-\alpha) * (Tasks(l)/T)$
        \State $FT(l) \gets (r + t) * (d - CP)$
    \EndFor
    \For{$task\ t\ in\ w$}
        \State $est \gets 0$
        \For{p in Pred(t)}
            \State $est \gets Deadline(p)$
        \EndFor
        \State $Deadline(t) \gets est + RT(t) + FT(Level(t))$
    \EndFor
\EndProcedure
\end{algorithmic} 
\end{algorithm}

\section{Performance Evaluation}


\section{Conclusion and Future Work}


%\section{Acknowledgments}
%Acknowledgements go here.

\bibliographystyle{abbrv}
\bibliography{paper}

%\appendix
%\section{Headings in Appendices}

%\balancecolumns

\end{document}
